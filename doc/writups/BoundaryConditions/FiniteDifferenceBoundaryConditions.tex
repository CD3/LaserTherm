\documentclass[letterpaper,12pt]{article}

\usepackage{graphicx}
\usepackage{cancel}
\usepackage{units}
\usepackage{amsmath}
\usepackage{txfonts}
\usepackage{tensor}
\parindent=0in

\newcommand \bcf{f}



\title{Notes on the Boundary Conditions in Finite-Difference Algorithms}
\author{C.D. Clark III}


\begin{document}
\maketitle


\section{Implicit Algorithms}
Any implicit finite difference algorithm is based on getting a matrix equation of the form
\begin{equation}
\mathbb{A} \nu = b
\end{equation}
where, $\mathbb{A}$ is a known matrix, $b$ is a known vector, and $\nu$ is the vector of unknowns to be solved for.
For most common Finite-Difference methods, the matrix $\mathbb{A}$ is tridiagonal, and so only three elements of $\nu$ appear in each equation. The vector of knowns, $b$, must be calculated, and this to typically involves three elements of the "current" solution. In other words, in general we usually have something like this:
\begin{equation}
  \mathbb{A} \nu^{n+1} = \mathbb{C} \nu^{n} + d
\end{equation}
where $\mathbb{C}$ is a known matrix that operates on the current solution and gives a vector of constants, and $d$ is a vector of constants that do not depend on the current solution.

This matrix equation represents a system of equation. One equation in this system can be written
\begin{equation}
  \label{eqn:FD_eqn}
  a\indices{^L_i} \nu\indices{^{n+1}_{i-1}}
+ b\indices{^L_i} \nu\indices{^{n+1}_{i  }}
+ c\indices{^L_i} \nu\indices{^{n+1}_{i+1}}
=
  a\indices{^R_i} \nu\indices{^{n  }_{i-1}}
+ b\indices{^R_i} \nu\indices{^{n  }_{i  }}
+ c\indices{^R_i} \nu\indices{^{n  }_{i+1}}
+ d_i
\end{equation}
There are two "boundaries" that have to be considered here. When $i = 0$,
\ref{eqn:FD_eqn}, refers to $\nu\indices{_{-1}}$, which does not exist. At $i =
N-1$, $\nu\indices{_N}$ is referred to, which also does not exist. These two
equation, $i = 0$ and $i = N-1$, have to be replaced with equations that do not
refer to the non-existent elements of $\nu$. Actually, these two equation are
just modified to only refer to actual elements of $\nu$.  To write the
modification, a second equation relating the non-existent element of $\nu$ to
the existing elements is required. This equation is called the ``boundary
condition''. The boundary condition used is determined by the physics.

The simplest boundary condition to implement is the ``sink''. In this case, the
solution is required to be zero at the boundary, so for example if the element
$\nu\indices{_i}$ did exist, it would be zero. In this case, the terms in
\ref{eqn:FD_eqn} referring to $\nu\indices{_{-1}}$ and  $\nu\indices{_{N}}$ can
just be ignored. The $i=0$ equation is then
\begin{equation}
  \label{eqn:FD_sink_eqn}
  b\indices{^L_0} \nu\indices{^{n+1}_{0}}
+ c\indices{^L_0} \nu\indices{^{n+1}_{1}}
=
  b\indices{^R_0} \nu\indices{^{n  }_{0}}
+ c\indices{^R_0} \nu\indices{^{n  }_{1}}
+ d_0
\end{equation}
and the $i = N-1$ equation is similar,
\begin{equation}
  \label{eqn:FD_sink_eqn_2}
  a\indices{^L_0} \nu\indices{^{n+1}_{N-2}}
+ b\indices{^L_0} \nu\indices{^{n+1}_{N-1}}
=
  a\indices{^R_0} \nu\indices{^{n  }_{N-2}}
+ b\indices{^R_0} \nu\indices{^{n  }_{N-1}}
+ d_0
\end{equation}
However, in general, we could allow the boundary be specified as an arbitrary constant,
\begin{equation}
  \label{eq:sink_bc}
  \nu\left(x_0\right) = \bcf
\end{equation}
In which case we have
\begin{equation}
  a\indices{^L_0} \bcf
+ b\indices{^L_0} \nu\indices{^{n+1}_{0}}
+ c\indices{^L_0} \nu\indices{^{n+1}_{1}}
=
  a\indices{^R_0} \bcf
+ b\indices{^R_0} \nu\indices{^{n  }_{0}}
+ c\indices{^R_0} \nu\indices{^{n  }_{1}}
+ d_0
\end{equation}
The first term on the LHS side here is a constant, so it must be moved over the right right hand side,
\begin{equation}
  b\indices{^L_0} \nu\indices{^{n+1}_{0}}
+ c\indices{^L_0} \nu\indices{^{n+1}_{1}}
=
  b\indices{^R_0} \nu\indices{^{n  }_{0}}
+ c\indices{^R_0} \nu\indices{^{n  }_{1}}
+ d_0
+ a\indices{^R_0} \bcf
- a\indices{^L_0} \bcf
\end{equation}
If we define a $d'_0$ as
\begin{equation}
d'_0 = a\indices{^R_0} \bcf - a\indices{^L_0} \bcf
\end{equation}
then we can write the $i=0$ equation as
\begin{equation}
  b\indices{^L_0} \nu\indices{^{n+1}_{0}}
+ c\indices{^L_0} \nu\indices{^{n+1}_{1}}
=
  b\indices{^R_0} \nu\indices{^{n  }_{0}}
+ c\indices{^R_0} \nu\indices{^{n  }_{1}}
+ d_0
+ d'_0
\end{equation}
For the $i=N-1$ equation, we have
\begin{equation}
  d'_{N-1} = c\indices{^R_{N-1}} \bcf - c\indices{^L_{N-2}} \bcf
\end{equation}
\begin{equation}
  a\indices{^L_{N-1}} \nu\indices{^{n+1}_{N-2}}
+ b\indices{^L_{N-1}} \nu\indices{^{n+1}_{N-1}}
=
  a\indices{^R_{N-1}} \nu\indices{^{n  }_{N-2}}
+ b\indices{^R_{N-1}} \nu\indices{^{n  }_{N-1}}
+ d_{N-1}
+ d'_{N-1}
\end{equation}

The other type of boundary condition that can be specified involves the
derivative of $\nu$ (with respect to $x$) at the boundary. This is the type of
boundary condition is required to model surfaces where the material loses energy
to the environment through convection, radiation, and evaporation. It is also
the boundary condition used to model an insulator.  In general, any boundary
condition can be written as
\begin{equation}
  \label{eq:heat_flux_bc}
  \left. \frac{\partial \nu}{\partial x} \right|_{x = boundary} = \bcf(\nu)
\end{equation}
By finite differencing the derivative, we obtain a relationship for the non-existent element of $x$.
\begin{equation}
  \label{eqn:FD_BC}
  \frac{ \nu\indices{_{i+1}} - \nu\indices{_{i-1}} }{ x\indices{_{i+1}} - x\indices{_{i-1}} } = \bcf(\nu_i)
\end{equation}

Let us only consider the $i=0$ case for now. Later, the $i=N-1$ case can be derived in a very similar manner. For $i=0$, equation \ref{eqn:FD_BC} gives
\begin{equation}
  \nu\indices{_{-1}} = \nu\indices{_{1}} - \Delta x \bcf(\nu_0)
\end{equation}
For the RHS of \ref{eqn:FD_eqn}, this can be incorporated directly,
\begin{equation}
  \label{eqn:lhs}
  a\indices{^R_0} \nu\indices{_{-1}^{n  }}
+ b\indices{^R_0} \nu\indices{_{ 0}^{n  }}
+ c\indices{^R_0} \nu\indices{_{ 1}^{n  }}
+ d_0
=
  a\indices{^R_0} \left[\nu\indices{^{n}_{1}} - \Delta x \bcf(\nu\indices{^{n}_{0}})\right]
+ b\indices{^R_0} \nu\indices{_{ 0}^{n  }}
+ c\indices{^R_0} \nu\indices{_{ 1}^{n  }}
+ d_0
\end{equation}
For the LHS however, $\bcf\left(\nu\right)$ needs to be evaluated at $\nu\indices{_{0}^{n+1}}$, which isn't known. We approximate this by expanding $\bcf\left(\nu\right)$ in a Taylor series, and only keeping the first two terms.
\begin{equation}
  \bcf(\nu\indices{_{0}^{n+1}}) \approx \bcf (\nu\indices{_{0}^{n}})
                                      +  \bcf'(\nu\indices{_{0}^{n}}) \left( \nu\indices{_{0}^{n+1}} - \nu\indices{_{0}^{n}} \right)
\end{equation}
Now the LHS of \ref{eqn:FD_eqn} can be written as
\begin{align}
  \label{eqn:rhs}
  a\indices{^L_0} \nu\indices{_{-1}^{n+1}}
+ b\indices{^L_0} \nu\indices{_{0 }^{n+1}}
+ c\indices{^L_0} \nu\indices{_{ 1}^{n+1}}
&=
a\indices{^L_0} \left[ \nu\indices{_{1}^{n+1}}
                    - \Delta x \left(\bcf (\nu\indices{_{0}^{n}})
                               +     \bcf'(\nu\indices{_{0}^{n}}) \left( \nu\indices{_{0}^{n+1}} - \nu\indices{_{0}^{n}} \right) \right)
               \right] \nonumber \\
&+ b\indices{^L_0} \nu\indices{_{0  }^{n+1}}
 + c\indices{^L_0} \nu\indices{_{  1}^{n+1}}
\end{align}
Note that any terms that are constant (i.e. do not contain a $\nu\indices{^{n+1}}$) must be moved over to the RHS, so after
some rearranging,
\begin{align}
  \label{eqn:minLHS}
  LHS&: 
  \left[b\indices{^L_0} - a\indices{^L_0}\Delta x \bcf'(\nu\indices{_{0}^n}) \right] \nu\indices{_0^{n+1}}
+ \left[c\indices{^L_0} + a\indices{^L_0}                                             \right] \nu\indices{_1^{n+1}} \\
  \label{eqn:minRHS}
  RHS&:
  b\indices{^R_0}\nu\indices{_0^n}
+ c\indices{^R_0}\nu\indices{_1^n}
+ d_0 \nonumber \\
&+
\left[
- a\indices{^L_0} \Delta x \bcf'(\nu\indices{_0^n}) \nu\indices{^n_0}
+ a\indices{^R_0}                 \nu\indices{_1^n} 
- a\indices{^R_0} \Delta x \bcf (\nu\indices{_0^n})
+ a\indices{^L_0} \Delta x \bcf (\nu\indices{_0^n})
\right]
\end{align}

Comparing these to \ref{eqn:FD_sink_eqn}, we see that in fact they just contain \ref{eqn:FD_sink_eqn} plus some extra terms. Therefore, in general,
we can \emph{assume} that we have a sink boundary condition when we create the representation for the first and last equations, and then add an arbitrary
boundary condition on by adding on to the right terms. All the extra information we need is the boundary condition function and its derivative with
respect to $\nu$.

Let
\begin{align}
  b\indices{^{'L}_0} &\equiv          - a\indices{^L_0} \Delta x \bcf'(\nu\indices{_0^n}) \\
  c\indices{^{'L}_0} &\equiv \phantom{-}a\indices{^L_0} \\
  b\indices{^{'R}_0} &\equiv          - a\indices{^L_0} \Delta x \bcf'(\nu\indices{_0^n}) \\
  c\indices{^{'R}_0} &\equiv \phantom{-}a\indices{^R_0} \\
  d\indices{^{' }_0} &\equiv \phantom{-}a\indices{^L_0} \Delta x \bcf (\nu\indices{_0^n}) - a\indices{^R_0} \Delta x \bcf (\nu\indices{_0^n})
\end{align}

Then we can write the boundary equation as
\begin{equation}
  \left( b\indices{^L_0} + b\indices{^{'L}_0} \right) \nu\indices{^{n+1}_{0}}
+ \left( c\indices{^L_0} + c\indices{^{'L}_0} \right) \nu\indices{^{n+1}_{1}}
=
  \left( b\indices{^R_0} + b\indices{^{'R}_0} \right) \nu\indices{^{n  }_{0}}
+ \left( c\indices{^R_0} + c\indices{^{'R}_0} \right) \nu\indices{^{n  }_{1}}
+ d_0 + d\indices{^'_0}
\end{equation}

For the other boundary, $i = N-1$, equation \ref{eqn:FD_eqn} gives
\begin{equation}
  \nu\indices{_{N}} =  \nu\indices{_{N-2}}  + \Delta x \bcf(\nu_0).
\end{equation}
Again, the boundary condition can be incorporated directly into the RHS, but a Taylor Series approximation must
be used for the LHS. The left and right sides will look like this
\begin{align}
 LHS&:
  a\indices{^L_{N-1}} \nu\indices{_{N-2}^{n+1}}
+ b\indices{^L_{N-1}} \nu\indices{_{N-1}^{n+1}} \nonumber \\
&+
\left[
  c\indices{^L_{N-1}} \nu\indices{_{N-2}^{n+1}}
+ c\indices{^L_{N-1}} \Delta x \bcf (\nu\indices{_{N-1}^{n}})
+ c\indices{^L_{N-1}} \Delta x \bcf'(\nu\indices{_{N-1}^{n}})\left(  \nu\indices{_{N-1}^{n+1}} - \nu\indices{_{N-1}^{n}} \right)
\right] \nonumber \\
  RHS&:
  a\indices{^R_{N-1}} \nu\indices{_{N-2}^{n  }}
+ b\indices{^R_{N-1}} \nu\indices{_{N-1}^{n  }}
+ d_{N-1} \nonumber \\
&+ \left[
  c\indices{^R_{N-1}} \nu\indices{_{N-2}^n}                  % the extra stuff from the RHS
+ c\indices{^R_{N-1}} \Delta x \bcf(\nu\indices{_{N-1}^n})  % the extra stuff from the RHS
\right] \nonumber
\end{align}
and then we have to move the constant terms from the LHS to the RHS
\begin{align}
  \label{eqn:maxLHS}
 LHS&:
  a\indices{^L_{N-1}} \nu\indices{_{N-2}^{n+1}}
+ b\indices{^L_{N-1}} \nu\indices{_{N-1}^{n+1}} \nonumber \\
&+
\left[
  c\indices{^L_{N-1}} \nu\indices{_{N-2}^{n+1}}
+ c\indices{^L_{N-1}} \Delta x \bcf'(\nu\indices{_{N-1}^{n}}) \nu\indices{_{N-1}^{n+1}}
\right] \\
  \label{eqn:maxRHS}
  RHS&:
  a\indices{^R_{N-1}} \nu\indices{_{N-2}^{n  }}
+ b\indices{^R_{N-1}} \nu\indices{_{N-1}^{n  }}
+ d_{N-1} \nonumber \\
&+ \left[
  c\indices{^R_{N-1}} \nu\indices{_{N-2}^n}                  % the extra stuff from the RHS
+ c\indices{^L_{N-1}} \Delta x \bcf'(\nu\indices{_{N-1}^{n}}) \nu\indices{_{N-1}^{n}}
+ c\indices{^R_{N-1}} \Delta x \bcf(\nu\indices{_{N-1}^n})  % the extra stuff from the RHS
- c\indices{^L_{N-1}} \Delta x \bcf (\nu\indices{_{N-1}^{n}})
\right]
\end{align}
Now, defining a set of prime constants again,

\begin{align}
  a\indices{^{'L}_{N-1}} &\equiv    c\indices{^L_{N-1}} \\
  b\indices{^{'L}_{N-1}} &\equiv    c\indices{^L_{N-1}} \Delta x \bcf'(\nu\indices{_0^n}) \\
  a\indices{^{'R}_{N-1}} &\equiv    c\indices{^R_{N-1}} \\
  b\indices{^{'R}_{N-1}} &\equiv    c\indices{^L_{N-1}} \Delta x \bcf'(\nu\indices{_0^n}) \\
  d\indices{^{' }_{N-1}} &\equiv    c\indices{^R_{N-1}} \Delta x \bcf(\nu\indices{_{N-1}^n}) - c\indices{^L_{N-1}} \Delta x \bcf (\nu\indices{_{N-1}^{n}})
\end{align}

Then we can write the boundary equation as
\begin{align}
  \left( a\indices{^L_{N-1}} + a\indices{^{'L}_{N-1}} \right) \nu\indices{^{n+1}_{N-2}}
+ \left( b\indices{^L_{N-1}} + b\indices{^{'L}_{N-1}} \right) \nu\indices{^{n+1}_{N-1}} \nonumber \\
=
  \left( a\indices{^R_{N-1}} + a\indices{^{'R}_{N-1}} \right) \nu\indices{^{n  }_{N-2}}
+ \left( b\indices{^R_{N-1}} + b\indices{^{'R}_{N-1}} \right) \nu\indices{^{n  }_{N-1}}
+ d_{N-1} + d\indices{^'_{N-1}}
\end{align}

So, we now have a general, concise method for implementing any type of Nuemann boundary condition. Note that, while in general the
method relies on an approximation involving the Taylor Series expansion of the boundary condition function, it gives the same
equations as directly implementing a boundary condition that is linear in $\nu$. Meaning, the boundary condition used to model energy loss
due surface convection,
\begin{align}
  \kappa \left.\frac{\partial \nu}{\partial x}\right|_{x=0} = h_e \left(\nu - \nu_{\infty}\right)
\end{align}
can be directly implemented without the use of a Taylor Series expansion, because there are no non-linear terms. However, doing
so leads to the exact same equations as \ref{eqn:minLHS}, \ref{eqn:minRHS}, \ref{eqn:maxLHS}, and \ref{eqn:maxRHS} with
$\bcf(\nu\indices{_{i}^n}) = \frac{h_e}{\kappa} \left(\nu\indices{_i^n} - \nu_{\infty}\right)$ and 
$\bcf'(\nu\indices{_{i}^n}) = \frac{h_e}{\kappa}$. The same is true for the insulating boundary conditions.
Therefore, there is no reason to implement linear boundary conditions separatly, this general method will give the exact same
results.

\subsection{Time-Dependent Boundary Conditions}

If the boundary condition is time dependent, for example if the ambient temperature is a function of time, then
the boundary conditions will be
\begin{equation}
  \label{eq:td_sink_bc}
  \nu\left(x_0\right) = \bcf(t)
\end{equation}
or
\begin{equation}
  \label{eq:td_heat_flux_bc}
  \left. \frac{\partial \nu}{\partial x} \right|_{x = boundary} = \bcf(\nu,t)
\end{equation}
, then the primed coefficients need to be updated. Any factors of $\bcf$ or $\bcf^\prime$ that start out on the left
hand side should be evaluated at the next timestep. So, for the sink boundary condition, we have
\begin{equation}
  d'_0 = a\indices{^R_0} \bcf(t) - a\indices{^L_0} \bcf(t + \Delta t)
\end{equation}
and
\begin{equation}
  d'_{N-1} = c\indices{^R_{N-1}} \bcf(t) - c\indices{^L_{N-2}} \bcf(t+\Delta t)
\end{equation}
For the heat flux boundary conditions we have
\begin{align}
  b\indices{^{'L}_0} &\equiv          - a\indices{^L_0} \Delta x \bcf'(\nu\indices{_0^n},t+\Delta t) \\
  c\indices{^{'L}_0} &\equiv \phantom{-}a\indices{^L_0} \\
  b\indices{^{'R}_0} &\equiv          - a\indices{^L_0} \Delta x \bcf'(\nu\indices{_0^n},t+\Delta t) \\
  c\indices{^{'R}_0} &\equiv \phantom{-}a\indices{^R_0} \\
  d\indices{^{' }_0} &\equiv \phantom{-}a\indices{^L_0} \Delta x \bcf (\nu\indices{_0^n},t+\Delta t) - a\indices{^R_0} \Delta x \bcf (\nu\indices{_0^n},t)
\end{align}
and
\begin{align}
  a\indices{^{'L}_{N-1}} &\equiv    c\indices{^L_{N-1}} \\
  b\indices{^{'L}_{N-1}} &\equiv    c\indices{^L_{N-1}} \Delta x \bcf'(\nu\indices{_0^n},t+\Delta t) \\
  a\indices{^{'R}_{N-1}} &\equiv    c\indices{^R_{N-1}} \\
  b\indices{^{'R}_{N-1}} &\equiv    c\indices{^L_{N-1}} \Delta x \bcf'(\nu\indices{_0^n},t+\Delta t) \\
  d\indices{^{' }_{N-1}} &\equiv    c\indices{^R_{N-1}} \Delta x \bcf(\nu\indices{_{N-1}^n},t) - c\indices{^L_{N-1}} \Delta x \bcf (\nu\indices{_{N-1}^{n}},t+\Delta t)
\end{align}
If the boundary conditions are constant in time, these primed coefficients reduce to the same as before. So the only terms that must be evaulated
for the current and next time time are the $d^{\prime}$'s. In fact, all of the primed coefficnets can be evaluted for the next time step \emph{except} the
$d^{\prime}$'s.



\end{document}




