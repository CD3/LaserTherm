\documentclass[letterpaper,12pt]{article}

\usepackage{graphicx}
\usepackage{cancel}
\usepackage{siunitx}
\usepackage{amsmath}
\usepackage{txfonts}
\usepackage{tensor}
\parindent=0in



\title{Analytical solutions to the Heat Equation for validating numerical solvers}
\author{C.D. Clark III}


\begin{document}
\maketitle

\section{Introduction}
\label{sec:intro}
This document provides a number of Heat Equation solutions that can be used to validate numerical heat solvers. 

\section{Eigen Modes}
The homogeneous heat equation can be written generally (without reference to a specific coordinate system) as:
\begin{equation}
  \label{eq:hom_heat_eq}
  \rho c \partial_t T = \nabla \cdot k \nabla T.
\end{equation}
For a material with uniform properties (actually, only the conductivity needs to be uniform), this can be simplified:
\begin{equation}
  \label{eq:simp_heat_eq}
  \rho c \partial_t T = k \nabla^2 T.
\end{equation}

This is similar to the wave equation, except that the time derivative is first order rather than second order. The eigen functions of
the $\nabla^2$ operator are also useful here. We can in fact determine the general solution by transforming to the eigen basis of
the Laplace operator, but here we want to find a specific solution suitable for comparing numerical heat solvers to.

The eigen functions of the Laplace operator have the interesting property that they decay in time without changing shape. So, if the initial
temperature distribution is an eigen function of $\nabla^2$, then the temperature distribution will not change shape in time, but simply
scale instead. These special eigen temperature distributions can be used to validate a heat solver in much the same way that a beam propagator
can be validated by propagating the eigen mode of a waveguide. We note that the eigen function equation for the Laplace operator is typically
written as
\begin{equation}
  \nabla^2 \phi = -\lambda^2 \phi
\end{equation}

Consider a temperature distribution which is an eigen function of the Laplace operator with eigenvalue $\lambda$.
The time-evolution of this temperature distribution is given by equation \ref{eq:simp_heat_eq}:
\begin{align}
  \rho c \partial_t \phi &= k \nabla^2 \phi = -k \lambda^2 \phi, \\
         \partial_t \phi &= \frac{-k \lambda^2}{\rho c} \phi.
\end{align}
This is just an ordinary first-order differential equations in time, which has
a simple exponential solution in time. Let $\alpha = \frac{k \lambda^2}{\rho c}$, then
\begin{equation}
  \phi(t,\vec{r}) = e^{-\alpha t} \phi(0,\vec{r}).
\end{equation}

\subsection{1D Cartesian Coordinates}
In one-dimensional Cartesian coordinates, equation \ref{eq:simp_heat_eq} becomes:
\begin{equation}
  \label{eq:1d_simp_heat_eq}
  \rho c \partial_t T = k \partial_{xx} T.
\end{equation}
The eigen equation for the conduction operator is
\begin{equation}
  \label{eq:1d_eigen_eq}
  \partial_{xx} \phi = -\lambda^2 \phi,
\end{equation}
which has solutions
\begin{equation}
  \phi \propto \sin(\lambda x), \cos(\lambda x).
\end{equation}
The eigenvalues, $\lambda$, are determined by the boundary conditions.


\subsubsection{Dirichlet Boundary Conditions}
Consider a uniform material over the domain $(0,L)$ with boundary conditions $T(0) = T(L) = 0$. The eigen functions satisfying these boundary
conditions are $A \sin(\lambda_m x)$ where $\lambda_m = \frac{m\pi}{L}$. So, for an initial temperature distribution $T(0,x) = A \sin(\frac{m\pi}{L} x)$
we will have
\begin{equation}
T(t,x) = e^{-\alpha t} A \sin\left(\frac{m\pi}{L} x\right) = 
e^{-\frac{k m^2\pi^2}{\rho c L^2} t} A \sin\left(\frac{m\pi}{L} x\right)
\end{equation}

\subsubsection{Neumann Boundary Conditions}
Consider a uniform material over the domain $(0,L)$ with boundary conditions
$\left.\partial_x T \right|_{x = 0} = \left.\partial_x T \right|_{x = L}= 0$.
The eigen functions satisfying these boundary
conditions are $A \cos(\lambda_m x)$ where $\lambda_m = \frac{m\pi}{L}$. So, for an initial temperature distribution $T(0,x) = A \cos(\frac{m\pi}{L} x)$
we will have
\begin{equation}
T(t,x) = 
e^{-\frac{k m^2\pi^2}{\rho c L^2} t} A \cos\left(\frac{m\pi}{L} x\right)
\end{equation}

\subsection{2D Cylindrical Coordinates}
For problems exhibiting azimuthal symmetry, we work in two-dimensional cylindrical coordinates, $r$ and $z$. The eigen
equation is
\begin{equation}
  \frac{1}{r} \partial_r (r \partial_r \phi) + \partial_{zz} \phi = -\lambda^2 \phi.
\end{equation}
The eigenfunctions $\phi(r,z)$ can be written as the product of a function of $r$ and a function of $z$, $\phi(r,z) = \phi_r(r) \phi_z(z)$.
The eigen equation can be rewritten
\begin{equation}
  \phi_z \frac{1}{r} \partial_r (r \partial_r \phi_r) + \phi_r \partial_{zz} \phi_z =
  -\lambda^2 \phi_r\phi_z
\end{equation}
The above equation implies that
\begin{align}
 \label{eq:r_eigen_eq}
 \frac{1}{r} \partial_r (r \partial_r \phi_r)&= -\lambda_r^2 \phi_r \\
 \label{eq:z_eigen_eq}
                        \partial_{zz} \phi_z &= -\lambda_z^2 \phi_z
\end{align}
Equation \ref{eq:z_eigen_eq} is just equation \ref{eq:1d_simp_heat_eq}, so the eigen functions $\phi_z$ will also be sine and
cosine functions. Equation \ref{eq:r_eigen_eq} can be rewritten as
\begin{equation}
 \partial_{rr} \phi_r + \frac{1}{r} \partial_r \phi_r = -\lambda_r^2 \phi_r.
\end{equation}
Multiplying by $r^2$ and substituting $\rho  = \lambda_r r$ shows that this is just Bessel's differential equation with $\alpha = 0$,
\begin{align}
 r^2\partial_{rr} \phi_r + r \partial_r \phi_r = -r^2 \lambda_r^2 \phi_r \\
 \frac{\rho^2}{\lambda^2} \partial_{\rho\rho} \phi_r + \frac{\rho}{\lambda} \partial_\rho \phi_r = -\rho^2 \phi_r \\
 \frac{\rho^2}{\lambda^2} \partial_{ r   r  } \phi_r + \frac{\rho}{\lambda} \partial_r    \phi_r + \rho^2 \phi_r = 0 \\
       \rho^2             \partial_{\rho\rho} \phi_r +       \rho           \partial_\rho \phi_r + \rho^2 \phi_r = 0
\end{align}
The solutions are Bessel functions, we require the temperature to be finite at $r=0$, so they are of the first kind,
\begin{equation}
 \phi_r(r \propto J_0(\rho) = J_0(\lambda_r r).
\end{equation}
Again, the eigenvalues are determined by the boundary conditions.

Once both $\lambda_z$ and $\lambda_r$ are set, the value of $\lambda$ will be $\lambda^2 = \lambda_z^2 + \lambda_r^2$.

\subsubsection{Dirichlet Boundary Conditions}
Consider a uniform material over the domain $(r,z) \in (0,R)\times(0,L)$ with boundary conditions $T(r,0) = T(r,L) = T(R,z) = 0$.
The $z$ eigen functions satisfying these boundary conditions are again 
$A \sin(\lambda_{zm} z)$ where $\lambda_{zm} = \frac{m\pi}{L}$.
The $r$ eigen functions will be Bessel function, $A J_0(\lambda_{rn} r)$ with $\lambda_{rn} = \frac{\alpha_n}{R}$ where
$\alpha_n$ is the $n$'th zero of $J_0(\rho)$.
\begin{equation}
T(t,r,x) = e^{-\alpha t} A \sin\left(\frac{m\pi}{L} z\right) J_0\left(\frac{\alpha_n}{R} r\right) = 
e^{-\frac{k}{\rho c}\left( \frac{m^2\pi^2}{L^2} + \frac{\alpha_n^2}{R^2}\right) t} A \sin\left(\frac{m\pi}{L} z\right) J_0\left(\frac{\alpha_n}{R} r\right)
\end{equation}

In order to compare this solution to a numerical solution, we will need to have a value for $\alpha_n$. An analytical expression
for these zeros does not exist, but there are several numerical libraries that provide functions to calculate them.

\subsubsection{Neumann Boundary Conditions}
Consider a uniform material over the domain $(r,z) \in (0,R)\times(0,L)$ with boundary conditions
$\left.\partial_z T\right|_{(r,0)} = \left.\partial_z T\right|_{(r,L)} = \left.\partial_r T\right|_{(R,z)} = 0$.
The $z$ eigen functions satisfying these boundary conditions are
$A \cos(\lambda_{zm} z)$ where $\lambda_{zm} = \frac{m\pi}{L}$.
The $r$ eigen functions will again be Bessel function, $A J_0(\lambda_{rn} r)$, but now the derivative must
be zero $r = R$. This means that one of the peaks (or valleys) must be at the boundary. Let $\lambda_{rn} = \frac{\beta_n}{R}$ where
$\beta_n$ is the coordinates of the $n$'th peak or valley of $J_0(\rho)$. This gives
\begin{equation}
T(t,r,x) = e^{-\alpha t} A \cos\left(\frac{m\pi}{L} z\right) J_0\left(\frac{\alpha_n}{R} r\right) = 
e^{-\frac{k}{\rho c}\left( \frac{m^2\pi^2}{L^2} + \frac{\beta_n^2}{R^2}\right) t} A \sin\left(\frac{m\pi}{L} z\right) J_0\left(\frac{\beta_n}{R} r\right)
\end{equation}
The zero's $\beta_n$ are again not known analytically, and numerical libraries do not calculate them, but we can determine them by using
the properties of Bessesl functions. The derivative of $J_0(\rho)$ is itself a Bessel function,
\begin{equation}
  \partial_\rho J_0(\rho) = -J_1(\rho)
\end{equation}
So, the peaks and valleys of $J_0$ can be determined by computing the zeros of $J_1$.
\end{document}


