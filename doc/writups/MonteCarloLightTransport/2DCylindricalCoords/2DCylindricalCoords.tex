\documentclass[letterpaper,12pt]{article}

\usepackage{graphicx}
\usepackage{cancel}
\usepackage{siunitx}
\usepackage{amsmath}
\usepackage{txfonts}
\usepackage{tensor}

\parindent=0.5in


\newcommand{\func}[1]{\left(#1\right)}

\title{Notes on Monte Carlo simulation for light transport in 2D cylindrical coordinates.}
\author{C.D. Clark III}


\begin{document}
\maketitle

\section{Introduction}
Monte Carlo simulations, such as the Monte Carlo Multiple Layer (MCML) model, are very common for simulating the distribution of absorbed power in turbid media (such as tissue).
The concept is quite simple. At it most basic level, the simulation is based on propagating each photon through a random walk. Photons travel between interaction sites where they can
be absorbed or scattered. The probability of absorption, scattering, and the direction of scattering are related to the media properties (absorption coefficient, scattering coefficient, anisotropy).

The most direct way to program a MC simulation is to use 3-dimensional Cartesian coordinates. A full 3D simulation would store the absorbed photon power in a 3D grid, but we often only need the absorbed
power on a 2D grid that corresponds to a slice out of a cylinder, i.e. cylindrical coordinates. If the system has azimuthal symmetry, then we can use this to determine the absorbed power in the 2D grid and speed
up the simulation by requiring fewer photons. However, we must be careful to do this mapping correctly.

\section{The Problem}
We want to compute the power density function in 2D cylindrical coordinates, $A(r,z)$, using a 3D Cartesian coordinate Monte Carlo simulation. Our simulation yields an absorbed power grid, $P_{i,j}$, corresponding
to the total photon power absorbed within the region $( i\Delta r < r < (i+1)\Delta r, i\Delta z < z < (j+1)\Delta z)$. Our problem is to obtain $A(r,z)$ from $P_{i,j}$.

\section{Derivation}

Assume that we have computed the photon absorbed power density, $A$, in 3D, Cartesian coordinates, $A \rightarrow A(x,y,z)$. The dimensions of $A$ are power per unit volume. In SI units, we have \si{\watt\per\meter\cubed}.
We could also write $A$ in cylindrical coordinates, $A \rightarrow A(r,\theta,z)$. The dimensions are the same, they do not depend on the coordinates used.

Now, if the function $A$ is azimuthally symmetric, then $A$ will not depend on $\theta$ and
\begin{align*}
  A(r,\theta,z) &= A(r,\phi,z) \\
  A(r,\theta,z) &\rightarrow A(r,z).
\end{align*}
This means that the average of $A$ over $\theta$ will be equal to $A$ at any $\theta$. Consider a line along constant $r$ and $z$. It will be a circle. The average power density over the line is
\begin{align}
  \label{eq:avg_A}
  \bar{A}(r,z) &= \frac{1}{2\pi r} \int\limits_0^{2\pi}  A(r,\theta,z) \; r d\theta \nonumber \\
               &= \frac{1}{2\pi} \int\limits_0^{2\pi}  A(r,\theta,z) \; d\theta \nonumber \\
               &= A(r,z)
\end{align}
In other words, just add up $A$ along the line and then divide by the length of the line.
\textbf{Note that the dimensions of $A(r,z)$ are still power per unit volume.}

Simulations in 2D cylindrical coordinates usually perform Monte Carlo
simulation in 3D Cartesian coordinates, but ``score'' photons onto a two
dimensional grid. When a photon is absorbed, its absorbed power is added to
the grid by mapping the photon's Cartesian coordinates to cylindrical
coordinates: $r = \sqrt{x^2 + y^2}$, $z = z$. 

Let a superscript $i$ denote the parameters corresponding to the $i$'th absorption event. Meaning, $P^i(r^i,\theta^i,z^i)$ is the power absorbed
at $(r^i,\theta^i,z^i)$ during the $i$'th absorption event. The total absorbed power then is just the sum of $P^i$ over all $i$. The total power
must also correspond to the integral of $A$. Therefore,
\begin{equation}
  \int_{0}^{2\pi}
  \int_{0}^{\infty}
  \int_{-\infty}^{\infty}
  A(r,\theta,z)
  r d\theta
  d r
  d z \approx \sum_i P^i
\end{equation}

This is the connection between the power density $A$ that we want
to compute and the scoring grid that we have.  Consider the total power
absorbed in a ring of width $\Delta r$ and thickness $\Delta z$ at $(r_i =
i\Delta r, z_j = j\Delta z)$, When we score photon power on a 2D grid by binning
power according to $r = \sqrt{x^2 + y^2}$, this is exactly what we are
computing.  Therefore, the relationship between the power density $A$
that we want to compute and the scoring grid that we have is
\begin{equation}
  dP = \int_{0}^{2\pi}
       A( r_i, \theta, z_j)
       r_i d\theta
       \Delta r
       \Delta z = P_{i,j}.
\end{equation}
Using Equation \ref{eq:avg_A} gives
\begin{equation}
       P_{i,j} =
       2\pi r_i A(r_i,z_j)
       \Delta r
       \Delta z
\end{equation}
which is the relationship we are looking for,
\begin{equation}
  A(r_i,z_j) = \frac{1}{2\pi \Delta r \Delta z r_i } P_{i,j}.
\end{equation}
The factor $2\pi\Delta r \Delta z$ can often be ignored because the function $A(r,z)$ will be normalized to give a specific power and these constants will be have no effect on the normalization. The
factor $r_i$ however must be included as it depends on the radial position of the bin. So, to correctly compute the absorbed photon power distribution on a 2D cylindrical grid from a 3D Cartesian simulation,
each bin weight should be divided by its radial position.

\end{document}




